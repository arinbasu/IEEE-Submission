\textit{Extended Abstract} 

In an age of pervasive and omnipresent media and recording of the self with The purpose of this paper is to provide a brief survey of the literature in the form of a systematic review and meta analysis of the efficacy and effectiveness of existing technology and wearable computing tools and tools and technology pertaining to the "Internet of Things" or "Internet of Me" tools for learning and teaching effectiveness. This systematic review will also form the basis of framing a larger discussion on how video self modelling as a tool can be effectively used and blended with Internet of Things to develop novel teaching and learning approaches in a range of settings. 

Video self modelling refers to ... 

The organisation of the paper will be as follows. We shall provide a brief background history of the emergence of the Internet of Things (wikipedia entry here: http://en.wikipedia.org/wiki/Internet_of_Things), and the concept of video self modelling and how their potential in modifying teaching and learning practices in tertiary sector and professional education (??). We shall then provide a systematic review of the effectiveness and efficacy of the available tools for educational practices, and finally, we shall provide a synthesis of how the more effective technologies can be blended with the principles and practices of video self modelling to foster more effective learning and can augur emergence of effective, cost-efficient learnign tools.

The systematic review should focus on the following question, "\textit{Compared with traditional approaches, what is the overall effectiveness of wearable technology and Internet of Things based applications in fostering teaching and learnign for tertiary students and students at professional degrees (nursing, health, education)}?" We shall focus on English language interventional studies published in the last 10 years. We shall search electronic databases (ERIC, Medline/Pubmed/UC library databases), and in addition,  hand search other databases, reference lists of he original documents, and search for fugitive literature by contacting domain experts and authors in the field. In the review, We shall critically appraise information from the literature and identify and rank order the publications on the risk of biases, and estimate summary effect sizes for different combinations of technological tools and learning/teaching outcome measures. 

We shall then discuss the results and implications of the systematic review in the light of the video self modelling as a pedagogical tool to enhance technology based learning and teaching practices. 

At the time of writing this abstract, we were unable to identify any summary of the available evidence on the effectiveness of the available tools. Our review will be the first systematic appraisal of key liteature of this emergent "game changing" technology. 