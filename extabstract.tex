\textit{Extended Abstract} 

In an age of the "Internet of Things" and "Internet of Self", pervasive and omnipresent electronic media and recording of the self at the physical and biochemical levels, these tools, particularly those that enable real time monitoring of self, become powerful instruments of education and training. While their implementation is new, they are by no means novel. Lower fidelity technology using self videos have been used to train and teach individuals for behavioural modelling and fostering learning (video self modelling). Therefore there is now an opportunity to compare and contrast how the two technologies, one novel, and the other, established but similar to each other in themes. 

Despite several implementations of the self measuring technologies, it is not clear how the two different approaches compare and contrast with each other. Hence the objective of this paper is to provide a comprehensive survey of the literature in the form of a systematic review and meta analysis of the effectiveness of wearable computing, "Internet of Things" for learning and teaching effectiveness, and compare the results, and the technological implementations with video self modelling. 

The organisation of the paper will be as follows. We shall provide a brief background history of the emergence of the Internet of Things (wikipedia entry here: http://en.wikipedia.org/wiki/Internet_of_Things), and the concept of video self modelling and how their potential in modifying teaching and learning practices in tertiary sector and professional education (??). We shall then provide a systematic review of the effectiveness and efficacy of the available tools for educational practices, and finally, we shall provide a synthesis of how the more effective technologies can be blended with the principles and practices of video self modelling to foster more effective learning and can augur emergence of effective, cost-efficient learning tools.

The systematic review should focus on the following question, "\textit{Compared with traditional approaches, what is the overall effectiveness of wearable technology and Internet of Things based applications in fostering teaching and learnign for tertiary students and students at professional degrees (nursing, health, education)}?" We shall focus on English language interventional studies published in the last 10 years. We shall search electronic databases (ERIC, Medline/Pubmed/UC library databases), and in addition,  hand search other databases, reference lists of he original documents, and search for fugitive literature by contacting domain experts and authors in the field. In the review, We shall critically appraise information from the literature and identify and rank order the publications on the risk of biases, and estimate summary effect sizes for different combinations of technological tools and learning/teaching outcome measures. 

We shall then discuss the results and implications of the systematic review in the light of the video self modelling as a pedagogical tool to enhance technology based learning and teaching practices. 

At the time of writing this abstract, we were unable to identify any summary of the available evidence on the effectiveness of the available tools. Our review will be the first systematic appraisal of key liteature of this emergent "game changing" technology. 