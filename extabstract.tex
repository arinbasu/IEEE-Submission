\subsection{Extended Abstract}


We are living in an age of the “Internet of Things”(IoT) and “Internet of Self”(IoS), pervasive and omnipresent electronic media and “quantified self” at the physical and biochemical levels. In turn, these tools, particularly those that enable real time monitoring of self in the form of glasses, watches, wearable computing devices, and fabrics can these be harnessed for education?

The use of video is now ubiquitous in learning situations including the use of short video sequences demonstrating practice exercises and instructional sequences (eg Khan Academy) allowing anytime anywhere study. Rapid learning from the future and its associated self model theory (Dowrick 2012) draws on over four decades of self modeling using purposefully constructed videos. These have been used to teach, through observation, individuals skills and procedures that they are not able to achieve at present, but by viewing a possible future depicting the individual successfully engaged in achieving, rapid learning takes place usually within several viewings. These Feedforward video clips are typically less than two minutes in length (Dowrick 1976). Thus there is now an opportunity to merge through comparison and contrasting how the two technologies compare with each other in furthering education.

Despite several implementations of the self measurement technologies, it is not clear how self modeling and emergent technologies based on Internet of Things, wearable computing, and personal monitoring devices compare and contrast with each other to further education. It is clear that the internet of things and the internet of self (personalized learning opportunities) play and will continue to play huge part in the developing world. We need to be clear we are talking about the opportunity to learn. To make an opportunity become a learning experience we need the individual to engage in a cognitive process, wearing or carrying a device provides the opportunity, cognition the learning. Hence the objective of this paper is to provide a comprehensive survey of the literature in the form of a systematic review and meta analysis of the effectiveness of wearable computing and “Internet of Things” for learning and teaching effectiveness, and compare and contrast the results, and the technological implementations with what is known about the effectiveness of self modeling. Such a comparison will be helpful in predicting the potential success and may provide new directions for the use of the emergent tools.

The organization of the paper will be in three parts as follows. We shall first provide a brief background history of the emergence of the Internet of Things, the concept of video self modeling and describe their potential in changing teaching and learning practices in the tertiary sector particularly in professional education. Next, we shall conduct a systematic review of the effectiveness of the emergent technologies (Internet of things, and wearable computing devices) for educational purposes, and finally, we shall provide a synthesis of how the more effective technologies can be compared, contrasted, and combined with the practices of self modeling to foster more effective learning, thereby creating effective, cost-efficient learning tools.

The systematic review will focus on the following question, “Compared with traditional approaches, what is the overall effectiveness of wearable technology and Internet of Things based applications in fostering teaching and learning in professional degrees (eg nursing, health, education)?” We shall select English language intervention studies published in the last 10 years. We shall search electronic databases (ERIC, Medline/Pubmed/University of Canterbury library databases), and in addition, hand search other databases, reference lists of the original documents, and search for fugitive literature by contacting domain experts, existing projects, and notable authors in the field. In the review, We shall critically appraise information from the literature to study possible biases to ascertain the quality of the studies, and rank order the publications on their assessed qualities. We shall estimate summary effect sizes for different combinations of technological tools and learning/teaching outcome measures wherever possible.

This process will lead to a discussion of the results and implications of the systematic review in the light of the self modeling as a pedagogical tool to enhance learning and teaching practices.
At the time of writing this abstract, no review of the effectiveness of IoT and wearable technology is available. As such, our proposed review will be the first systematic appraisal of key literature of this emergent “game changing” technology.
