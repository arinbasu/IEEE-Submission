\subsection{Extended Abstract}


We are living in an age of the "Internet of Things" and "Internet of Self", pervasive and omnipresent electronic media and "quantified self" at the physical and biochemical levels. In turn, these tools, particularly those that enable real time monitoring of self in the form of glasses, watches, wearable computing devices, and fabrics can be powerful instruments for education and training. 

In education and training, application of the self evaluation tools are by no means novel. The use of video is now ubiquitous in learning situationsincluding the use of short video sequences demonstrating practice exercises and instructional sequences (eg Khan Academy) allowing learners to study anytime anywhere where there is an internet connection.  Rapid learning from the future and its associated self model theory (Dowrick 2012) draws on over four decades of  self modeling using purposefully constructed videos. ("video selfies"). These have been used to teach, through observation, individuals skills and procedures that they are not able to achieve at present, but by viewing a possible future depicting the individual successfully engaged in achieving rapid learning takes place usually within several viewings. These Feedforward video clips are typically less than two minutes in length. Feedforward is the term that has been coined by Dowrick to describe this (Dowrick 1976). for behavioural modeling and fostering learning (video self modeling). Thus there is now an opportunity to compare and contrast how the two technologies compare and contrast with each other in education. 

Despite several implementations of the self measurement technologies, it is not clear how video self modelling and emergent technologies based on Internet of Things, wearable computing, and personal monitoring devices compare and contrast with each other in education. It is clear that the internet of things and the internet of self (personalized learning opportunities) play and will continue to grow a huge part in the developing world.  We need to be clear we are talking about the opportunity to learn. To make an opportunity become a learning experience we need the individual to engage in a cognitive process, wearing or carrying a device provides the opportunity.   Hence the objective of this paper is to provide a comprehensive survey of the literature in the form of a systematic review and meta analysis of the effectiveness of wearable computing and "Internet of Things" for learning and teaching effectiveness, and compare and contrast the results, and the technological implementations with what is known about the effectiveness of video self modeling. Such a comparison will be helpful in predicting the potential success and may provide new directions for the use of the emergent tools.

The organisation of the paper will be in three parts as follows. We shall first provide a brief background history of the emergence of the Internet of Things (wikipedia entry here: http://en.wikipedia.org/wiki/Internet_of_Things), and the concept of video self modeling and how their potential in modifying teaching and learning practices in tertiary sector and professional education (??). Next, we shall conduct a systematic review of the effectiveness of the emergent technologies (Internet of things, and wearable computing devices) for educational practices, and finally, we shall provide a synthesis of how the more effective technologies can be compared, contrasted, and combined with the practices of video self modeling to foster more effective learning, and can augur emergence of effective, cost-efficient learning tools.

The systematic review should focus on the following question, "\textit{Compared with traditional approaches, what is the overall effectiveness of wearable technology and Internet of Things based applications in fostering teaching and learnign for tertiary students and students at professional degrees (nursing, health, education)}?" We shall select English language interventional studies published in the last 10 years. We shall search electronic databases (ERIC, Medline/Pubmed/UC library databases), and in addition,  hand search other databases, reference lists of he original documents, and search for fugitive literature by contacting domain experts and authors in the field. In the review, We shall critically appraise information from the literature and identify and rank order the publications on the risk of biases, and estimate summary effect sizes for different combinations of technological tools and learning/teaching outcome measures. 

We shall then discuss the results and implications of the systematic review in the light of the video self modelling as a pedagogical tool to enhance technology based learning and teaching practices. 

At the time of writing this abstract, we were unable to identify any summary of the available evidence on the effectiveness of the available tools. Our review will be the first systematic appraisal of key liteature of this emergent "game changing" technology. 