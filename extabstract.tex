\textit{Extended Abstract} 

We are living in an age of the "Internet of Things" and "Internet of Self", pervasive and omnipresent electronic media and "quantified self" at the physical and biochemical levels. In turn, these tools, particularly those that enable real time monitoring of self in the form of watches, wearable computing devices, and fabrics can be powerful instruments of education and training. In education and training, application of the self monitoring tools are by no means novel. Video self modelling, using using self videos ("video selfies") have been used to train and teach individuals for behavioural modelling and fostering learning (video self modelling). Thus there is now an opportunity to compare and contrast how the two technologies, compare and contrast with each other in education. 

Despite several implementations of the self measurement technologies, it is not clear how the two different approaches compare and contrast with each other. Hence the objective of this paper is to provide a comprehensive survey of the literature in the form of a systematic review and meta analysis of the effectiveness of wearable computing and "Internet of Things" for learning and teaching effectiveness, and compare and contrast the results, and the technological implementations with what is known about the effectiveness of video self modelling. Such a comparison will be helpful in predicting the potential success and may provide new directions for the use of the emergent tools.

The organisation of the paper will be as follows. We shall provide a brief background history of the emergence of the Internet of Things (wikipedia entry here: http://en.wikipedia.org/wiki/Internet_of_Things), and the concept of video self modelling and how their potential in modifying teaching and learning practices in tertiary sector and professional education (??). We shall then provide a systematic review of the effectiveness and efficacy of the available tools for educational practices, and finally, we shall provide a synthesis of how the more effective technologies can be blended with the principles and practices of video self modelling to foster more effective learning and can augur emergence of effective, cost-efficient learning tools.

The systematic review should focus on the following question, "\textit{Compared with traditional approaches, what is the overall effectiveness of wearable technology and Internet of Things based applications in fostering teaching and learnign for tertiary students and students at professional degrees (nursing, health, education)}?" We shall select English language interventional studies published in the last 10 years. We shall search electronic databases (ERIC, Medline/Pubmed/UC library databases), and in addition,  hand search other databases, reference lists of he original documents, and search for fugitive literature by contacting domain experts and authors in the field. In the review, We shall critically appraise information from the literature and identify and rank order the publications on the risk of biases, and estimate summary effect sizes for different combinations of technological tools and learning/teaching outcome measures. 

We shall then discuss the results and implications of the systematic review in the light of the video self modelling as a pedagogical tool to enhance technology based learning and teaching practices. 

At the time of writing this abstract, we were unable to identify any summary of the available evidence on the effectiveness of the available tools. Our review will be the first systematic appraisal of key liteature of this emergent "game changing" technology. 