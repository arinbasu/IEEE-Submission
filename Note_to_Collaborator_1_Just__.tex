Note to Collaborator

1. Just double click on the text of the 'chunk' to start editing it
2. You will see simple options of editing the text: from left to right, these are cite, table, B, I, <>, .-, 1-, ", link, three levels of headers
3. Notes: "cite" is for inserting citations. Here are the rules for inserting citations:

Click next to the text where you'd like to insert citation
Click "cite" on the top
This wil open a citation window where you can search for references. You can insert from the search results, OR,
You can import citations from your personal reference manager such as  Endnote. Use the BibTeX format to import citations and stick to the textbox

Inserting tables is by clicking on the grid image and  filling in the cells of the table. The first or header row is for table headers

The  B and I are for bolding or italicising the text if you want. 

The "<>" is for inserting computer codes such as R or Stata codes. We may not always need it

The next two are for lists (number lists and bullet lists)

For quote, select the text you want to  quote and put quote marks by clicking on it

Similarly the link, just put the URL between curly brackets 

And then the three levels of headers will be shown as sections so just select t he text you want to set as header (for whatever headiing level you want) and click the respective symbol (first level header h1 and so on)

That's all about editing the text blocks.

We can also comment at each  level of the section or block

There is no fear of losing data as it is backed up tightly and we can easily restore  old copies and backups

If you want to make changes  please go ahead and make changes.  Both the old and new texts will be available all the time.

Really look forward to your collaboration and cooperation. I find it very easy and intuitive and producive way of writing papers, and hopefully  you will find too. Takes a few minutes to adjust but after that, it is easy and intuitive. 


